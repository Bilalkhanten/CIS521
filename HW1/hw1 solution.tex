\documentclass[12pt]{article}

%AMS-TeX packages
\usepackage{amssymb,amsmath,amsthm} 
%geometry (sets margin) and other useful packages
\usepackage[margin=1.25in]{geometry}
\usepackage{graphicx,ctable,booktabs}
\usepackage{enumerate}
\usepackage{url}
\usepackage{graphicx}
\usepackage{color}
\usepackage{float}
\usepackage{auto-pst-pdf}

\definecolor{gray}{rgb}{0.5,0.5,0.5}
\definecolor{black}{rgb}{0,0,0}

%Redefining sections as problems
\makeatletter
\newenvironment{problem}{\@startsection
  {section}
  {1}
  {-.2em}
  {-3.5ex plus -1ex minus -.2ex}
  {2.3ex plus .2ex}
  {
    \large\bf\noindent{Problem }
  }
}
\makeatother


%Fancy-header package to modify header/page numbering 
\usepackage{fancyhdr}
\pagestyle{fancy}
%\addtolength{\headwidth}{\marginparsep} %these change header-rule width
%\addtolength{\headwidth}{\marginparwidth}
\lhead{\small CIS 521: Intro to AI }
\chead{} 
\rhead{Boyang Zhang} 
\lfoot{Homework \# 2} 
\cfoot{} 
\rfoot{Page \thepage} 
\renewcommand{\headrulewidth}{.3pt} 
\renewcommand{\footrulewidth}{.3pt}
\setlength\voffset{-0.25in}
\setlength\textheight{648pt}

%%%%%%%%%%%%%%%%%%%%%%%%%%%%%%%%%%%%%%%%%%%%%%%

\begin{document}

\begin{problem}{}
	\color{gray}
	Consider the following problem: You have two jugs of water with capacities 4 and 13 liters. You also have an infinite supply of water. Can you use the two jugs to get exactly 2 liters of water? Cast this as a search problem: what is your state space, initial state, action space, goal condition, and costs of actions?\\\\
	\color{black}
           State Space: All integer combinations of 0-4 liters in
           small jug and 0-13 liters in large jug.\\
            Initial state: Two empty jugs\\
            Action space: Fill smaller jug, pour smaller jug into
            larger jug, pour out larger jug.\\
            Goal condition: Exactly 2 liters of water\\
            Cost of action: Number of actions to reach the goal\\
            Fill 4liter jug and pour into 13 liter jug.  Do this 3
            times until 13 liter jug has 12 liters and 4 liter jug is
            empty.\\
            Next fill the 4liter jug and pour 1 liter into the 13
            liter.  Empty the 13liter and then pour 3 liters into 13
            liter jug.\\
            Refill 4liter and pour into 13 liter (7 liters now)\\
            Refill 4liter and pour into 13 liters(11 liters now)\\
            Refill 4 liter and pour into 13 liter(13 liters now), and
            4 liter bottle still has 2 liters.\\
            Empty 13 liter and we now have 2 liters.
\end{problem}

\begin{problem}{}
	\color{gray}
Describe a search space in which iterative deepening search performs much worse than depth-first
search. In this situation, give big-O descriptions of the running time for iterative deepening and for
depth-first search. \\ \\
	\color{black}
	
        In the case where you have a search space that has a branching
        factor of 1, the DFS will perform in $O(n)$ runtime whereas the
        iterative deepening search will perform in $O(n^2)$.
\end{problem}

\begin{problem}{}
	\color{gray}
 Sometimes there is no good evaluation function for a problem, but there is a good comparison method:
a way to tell if one node is better than another, without assigning numerical values to either. Show that
this is enough to do a best - first search. (a) Describe in words and pseudo code how you would implement this method. (b) Show what the differences in time and memory would be (if any) compared to
a standard greedy search where you know the particular value associated with each node.\\
	\color{black}

\begin{description}
\item{a)} If you can only compare two node, for each nodes, compare
  them and then enqueue onto a priority queue.  The priority queue
  will implement the comparision method to evaluate whether a node is
  better than another.\\
\item{b)} So the difference between the two searches is essentially
  that greedy search is similar to a DFS so it is more memory
  efficient than a best first search which is similar to a BFS.
  However, the greedy search does not necessarily find the optimal path
  therefore the best first search has a better runtime.
\end{description}
\end{problem}

\begin{problem}{}
	\color{gray}
There are n people $0,1,...,n-1$. They all need to cross the bridge but they have just one flashlight.
A maximum of two people may cross at a time. It is nighttime, so someone must carry the single
flashlight during each crossing. They all have different speeds such that the time taken to cross the
bridge for person i is given by T(i) minutes. You may assume without loss of generality that for
$i < j, T(i) ≥ T(j)$. The speed of two people crossing the bridge is determined by the slower of the
two. You need to find the shortest amount of time in which all the people can cross the bridge.
As an example suppose there are 4 person and time taken $(T(i))$ are $1,2,5$ and $10 for i = 0, 1, 2, 3$
respectively. \\ \\
	\color{black}
\begin{enumerate}
 \item The sum of time that it takes for everyone to cross.  This is
   admissible because it does not account for the time it would take
   for someone to cross back and therefore is the shortest time that
   people can cross and it will never be longer.  This is definitely
   consistent because the sum of times for everyone to cross is less
   than the total amount of time following the regular constraints
   because I relaxed the constraint that people need to go back.\\
  \item The shortest amount of time it takes one person to cross.
    Admissible because if everyone took the same amount of time then
    that would be the fastest possible time and if everyone else
    crossed slower than the fastest person it's still an optimistic
    heuristic.  It's consistent because it is the fastest time for
    someone to cross and everyone will either be slower or at the same
    time.\\
\item The total number of people left to cross.  Admissible because it
  will never be more people than there totally are that need to cross.\\
\end{enumerate}
	
\end{problem}

\begin{problem}{}
	\color{gray}
Which of the following are admissible, given admissible heuristics $h1$ and $h2$?
	\color{black}
	\begin{enumerate}[i.]
		\item $h(n) = min(h1(n), h2(n))$ \\
                  Admissible 

		\item  $h(n) = w h1(n) + (1 − w)h2(n)$, where $0 \leq w \leq 1$\\
                  Admissible

		\item $h(n) = max(h1(n), h2(n))$\\
                  Admissible
	\end{enumerate}
\end{problem}

\begin{problem}{}
	\color{gray}
Which of the following are consistent, given consistent heuristics $h1$
and $h2$?\\\\
	\color{black}
\begin{enumerate}
       \item $h(n) = min(h1(n), h2(n))$ \\
         Consistent
         \item $h(n) = w h1(n) + (1 − w)h2(n)$, where $0 \leq w \leq
           1$\\
           Consistent
           \item  $h(n) = max(h1(n), h2(n)$\\
           Consistent
    
\end{enumerate}
\end{problem}

\begin{problem}{}
  \color{gray}
Assume we have a very large search space with a very large branching factor at most nodes, and we
do not have any heuristic function. What search method would be good
to use in this situation? Why?\\\\
\color{black}

The large search space rules out DFS because the search time would
take too long and it would spend too much time going down one path, therefore we can examine iterative deepening search or
BFS.  Since we know that this search space has a large branching
factor we can eliminate BFS since a large branching factor would make
the space complexity way too large since BFS puts every child node
onto the queue therefore although IDDFS has a slower runtime than BFS
it saves on the space complexity.

\end{problem}

\begin{problem}{}
    \color{gray}
    Suppose that your successor function is to choose the next available square (reading the board left-toright, up-to-down, like English) and write down a number in that square. What is the branching factor
of this successor function?\\\\
\color{black}
It will be 9 because for a given square with an empty state, there are
9 possible numbers that can fill that space.\\

\end{problem}
\begin{problem}{}
\color{gray}
 Will Depth First Search (DFS) be finite on the Sudoku problem? Let r be the number of empty
positions on the initial Sudoku board. What is the maximum search
depth of DFS in terms of r.\\\\
\color{black}
If there are $r$ empty spaces then you will reach a maximum search
depth of $r$ because it will only search for empty spaces and fill
them.
\end{problem}
\begin{problem}{}
\color{gray}
 What is the best-case run time complexity of DFS, in terms of r? What
 is the worst-case?\\\\
\color{black}
Best-case: $O(9^r)$
Worst-case: $O(9^r)$ where 9 is the branching factor.
\end{problem}
\begin{problem}{}
\color{gray}
 What are the best-case and worst-case run times for Breadth First
 Search (BFS)?\\\\
\color{black}
Best case:$O(r)$
Worst case: $O(9^r)$ where 9 is the branching factor.
\end{problem}

\begin{problem}{}
\color{gray}
 We saw in class that the (worst-case) space complexity of BFS is often quite large. In the case of
Sudoku, suppose you try to run BFS for a board with r = 40 on your laptop. Assume (optimistically)
that storing a state in memory requires only 1 byte. How much memory might your laptop need by
the time it reached the solution?\\\\
\color{black}
Since each position on the sudoku board has 9 possible values.  The
worst case space complexity will be $9^{40}$ bytes.
\end{problem}

\begin{problem}{}
\color{gray}
In simulated annealing, we use a local representation of the problem. For Sudoku, the local representation is a complete assignment to the $r$ originally empty positions on the board. What is the size of
this entire state space, in terms of $r$?\\\\
\color{black}
$9^r$
\end{problem}

\begin{problem}{}
\color{gray}
 What is a successor function you might use for simulated annealing?\\\\
\color{black}
Take two values and swap them.
\end{problem}
\end{document}
