\documentclass[11pt,letterpaper]{article}
\usepackage{amsmath,amsfonts,amssymb}
\usepackage{times}
\usepackage{graphicx,float}
\usepackage{url}
\newcommand{\Points}[1]{[#1 points]}

% In case you need to adjust margins:
\topmargin=-0.45in      %
\evensidemargin=0in     %
\oddsidemargin=0in      %
\textwidth=6.5in        %
\textheight=9.0in       %
\headsep=0.25in         %

\title{CIS521 HW3: Boyang Zhang} \date{}

\begin{document}
\maketitle

\vspace{-30pt}


\begin{description}
\item{2.1.1}\\
The constraints are as follows:
\begin{itemize}
\item $Revenue = 5x_{1} + 1.5x_{2} + x{3} + 2x{4} + 7x{5}$
\item $x_{2} \ge 10$
\item $x_4 \ge 15$
\item $x_1, x_2, x_3 , x_4, x_5 \le 24$
\item $8x_1 + 5x_2 + 3x_3 + 6x_4 + 10x_5 \le 400$
\end{itemize}\\
Since $x_5$ has the highest price, my intuition is to try and max it
and min all the other values.  So we have:\\
$x_5 = 24$\\
$x_4 = 15$\\
$x_2 = 10$\\

This leaves us with just $x_1$ and $x_3$ to try and evaluate.
If we plug everything into the power constraint.  We get:\\
$8x_1 + 50 + 3x_3 + 90 + 240 \le 400$\\
$8x_1 + 3x_3 \le 20$\\
Since $x_1$ is the highest price, I will max it and min $x_3$. So we
get from the power equation:\\
$x_1 = 2.5$\\
$x_3 = 0$\\

Then from our revenue function:\\
$Revenue = 12.5 + 15 + 0 + 30 + 168 = 225.5$ as the max revenue\\
\item{2.1.2}
The previous solution is not valid because we have $x_1 = 2.5$ and
with the new constraints, we cannot have non-integer numbers.  Still
we can follow the general intution from before and max $x_5$. So we
will have:\\
$x_5 = 24$\\
$x_4 = 15$\\
$x_3 = ?$\\
$x_2 = 10$\\
$x_1 = ?$\\

Now we need to re-evaluate the power function simplified to:\\
$8x_1 + 3x_3 \le 20$\\
We have options that satisfy this constraint.\\
First: $x_1 = 1, x_3 = 4$\\
This will produce $revenue = 222$.  However this is not the max
value. Rather:\\
$x_1 = 2, x_3 =0 $ will give a $revenue = 223$ which is higher, but we
need to look further.\\
Max is actually:\\
$x_5 = 22$\\
$x_4 = 15$\\
$x_3 = 0$\\
$x_2 = 10$\\
$x_1 = 5$\\
Which gives a max revenue of $224$.
\item{2.1.3}
Since in problem 1, $x_1 = 2.5$, we will round this to $x_1 = 3$.
This will increase the max revenue to $228$.  However this will
invalidate the power constraint set above.  Thus we would need to add
a slack of 4 to the total power in order to not invalidate the
constraint since:\\
$8(3) + 5(10) + 0 + 6(15) + 10(24) = 404 kW/hr$\\
\item{2.1.4}
Problem 1 is the easiest to solve because it's an Linear programming
problem so it can be solved in polynomial time.  However, question 2
is harder to solve because it's set up as an integer programming
problem and without relaxing constraints would need a complex
potentially exponential time algorithm to solve.\\\\
\end{description}
\begin{description}
\item{2.2.1}
  Let $m_i =$ number of meetings with person $i$ for people from $1, 2, ... i$\\
  Let $s_i =$ start of meeting\\
  Let $e_i =$ end of meeting\\

Maximize $\sum_{i = 1}^n x_i$ for $n$ people.\\
We must set the constraint that separate timeslots cannot overlap
between people. So:\\
For the randomly chosen person $p_i$, $s_i$ cannot be between $s_j$
and $e_j$ for a person $p_j$ and the tuple $(s_i, e_i)$ must be in the
set of available time slots, $S^{(av)}_i$.\\

\item{2.2.2}

In Sudoku we had 27 constraints that contained the numbers 1-9.  We
must go through the set of 27 constraints and check to see if each
variable satisfies the constraint or not through containing integers 1-9.  In order to do this let us
see what this would look like as an integer program:\\

Let $\delta_ij = 0 or 1$ where the number $j$ from $1-9$ is within one
of the 27 constraints, $i$.\\
Since the integer programming question is asking us to convert the
logical constraint of $0$ or $1$ into an arithmetic constraint. We can
say that the $\sum_i \sum_j \delta_ij \le 243$ because there are 27
constraints and 9 numerical constraints. \\
\end{description}

\begin{description}
\item{2.3.1}\\
 \begin{description}
\item{a)}2, 9, 9\\
\item{b)}
  \begin{tabular}{ l | c | r | }
  \hline                       
  N & \alpha & \beta \\
\hline
  A & 4 & \infty \\
  B & -\infty & 4 \\
  C & 4 & 3 \\
  D & 4 & 1 \\
  E & 4 & 4 \\
  F & 8 & 8 \\
  G & 6 & 6 \\
  H & 7 & 7 \\
  I & 3 & 3 \\
  J & N/A & N/A \\
  K & 1 & 1 \\
  L & N/A & N/A \\
  M & N/a & N/A \\
  
  \hline  
\end{tabular}
\end{description}
\item{2.3.2}
\begin{description}
\item{a)}
  The best option would be to go left (toward node B) because the sum
  of the expected value of the left path is $(0.8)(2) + (0.2)(1) =
  1.8$.  This is greater than the sum of the expected value of the
  right path which is $(0.6)(-2) + (0.4)(2) = -0.4$.
\item{b)}
If we only know the first six leaves then we need to evaluate the
seventh and eight nodes because the first 6 nodes are in distinct
subtrees from the 7th and 8th node.  Because of this, we do not know
the range of possible values in the 7th and 8th nodes.  This is
because, while we know that the right path is $1.8$, the left path is
$(0.6)(-2) + (0.4)*x$ where x is the smaller value of 7 and 8.\\
However if we have seen the first seven nodes then we do not have to
evaluate the 8th node because we know that $B = 1.8$ and C is
currently evaluated to $C = -1.2 + 0.4(2) = -0.4$.  Since our $\alpha$
value carries over from the $\beta$ B of $1.8$  We know that the $C =
-0.4$ is a smaller $\beta$ than the $\alpha$ therefore we do not need
to evaluate the 8th node.
\item{c)} The value range of the left-hand node is between 1.2 and 2.0
  (1.6 + or - 4).
\item{d)}  If we know the range of the leaf nodes is between [-2,2]
  then we can prune looking at node 6, 7, 8 because we know that when
  we reach node 5, we know that the expected value cannnot be greater
  than $(0.6)(-2) + (0.4)x$ and this value cannot be greater than the
  value of the left path which is 1.8.
\end{description}
\end{description}

\end{document}
